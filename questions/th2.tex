\Large \textbf{فرایندهای Ergodic Mean}
\large \textbf{(25 نمره)}

\normalsize \vspace{0.5cm}

\begin{enumerate}[label=(\alph*)]
	\item
	دو فرایند
	${X_1}\left( t \right)$
	و
	${X_2}\left( t \right)$
	\lr{Mean Ergodic}
	هستند و میانگین آن‌ها به ترتیب برابر با
	${\eta _1}$
	و
	${\eta _2}$
	هستند. در صورتی که تعریف کنیم:
	$$
	X\left( t \right) = {X_1}\left( t \right) + c{X_2}\left( t \right)
	$$
	به صورتی که c یک متغیر برنولی مستقل با احتمال
	$\frac{1}{2}$
	باشد. آیا
	$X\left( t \right)$
	\lr{Mean Ergodic}
	است؟ چرا؟
	\item 
	فرایند زیر را در نظر بگیرید:
	$$
	X\left( t \right) = a\cos \left( {\omega t} \right) + b\sin \left( {\omega t} \right) + c
	$$
	که در آن a و b در متغیر تصادفی uncorrelated هستند که میانگین آن‌ها برابر با صفر و واریانسی یکسان دارند و
	$\omega $
	ثابت می‌باشد. آیا
	$X\left( t \right)$
	\lr{Mean Ergodic}
	است؟ چرا؟
	\item
	در صورتی که A و 
	$\omega $
	ثابت و
	$\phi $
	یک متغیر تصادفی با توزیع یکنواخت بین
	$\left[ { - \pi ,\pi } \right]$
	باشد. آیا فرایند زیر
	\lr{Mean Ergodic}
	است؟ چرا؟
	$$
	X\left( t \right) = A\cos \left( {\omega t + \phi } \right)
	$$
\end{enumerate}


