\Large \textbf{ویژگی‌های اتوکورلیشن فرآیند WSS}
\large \textbf{(20 نمره)}

\normalsize \vspace{0.5cm}
براساس فرآیند تصادفی WSS در دامنه حقیقی به سوالات زیر پاسخ دهید.
\begin{enumerate}[label=(\alph*)]
	\item
اثبات کنید که که اتوکورلیشن این فرآیند زوج است.
	\item
اثبات کنید که تابع اتوکورلیشن دارای یک ماکزیمم در نقطه صفر است. یعنی به عبارت ریاضی:
$$
\forall \tau : R_{X}(0) \geqslant R_X(\tau)
$$
(راهنمایی: ابتدا برای اتوکورلیشن یک کران بالا بر اساس نامساوی کوشی-شوارتز پیدا کنید سپس نشان دهید که 
$ R_X(0) $
برابر با این کران است.)
	\item 
بر اساس این دو خاصیت با ذکر دلیل بیان کنید که کدام یک از توابع نمی‌تواند تابع اتوکورلیشن یک فرآیند WSS باشند.
$$
\begin{array}{l}
	{R_X}(\tau ) = exp(0.01 \times |\tau |) + 4\\
	{R_X}(\tau ) = exp( - |\tau  - 4|) + 9\\
	{R_X}(\tau ) = exp( - |\tau | + 4) - 25
\end{array}
$$

\end{enumerate}


