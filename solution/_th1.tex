\Large \textbf{درست نادرست}
\large \textbf{(۱۰ نمره)}

\normalsize \vspace{0.5cm}
درستی و نادرستی عبارات زیر را با ذکر دلیل بیان کنید.
\begin{enumerate}[label=(\alph*)]
	\item
دنباله نمونه‌های i.i.d. از یک متغیر تصادفی، تشکیل یک فرآیند SSS می‌دهند.
	\\
درست است. طبق تعریف i.i.d. می‌دانیم که هر نقطه از زمان مستقل از دیگری است و همچنین تمام نقاط ویژگی آماری یکسانی دارند پس این فرآیند یک فرآیند SSS است.
	\item
هر فرآیند تصادفی WSS یک فرآیند تصادفی SSS نیز می‌باشد.
	\\
نادرست است. برعکس این موضوع صحیح است یعنی هر فرآیند SSS یک فرآیند تصادفی WSS نیز می‌باشد.
	\item
 برای تساوی 
$ R_X $
 و
$ C_X $
 یک فرآیند WSS صفر بودن میانگین شرط لازم و کافی است.
 	\\
 درست است. چون برای فرآیند WSS می‌دانیم که:
$$
C_X(\tau) = R_X(\tau) - \eta^2
$$
در عبارت جبری بالا اگر و تنها اگر 
$ \eta = 0 $
باشد تساوی برقرار است.

	\item
یک فرآیند می‌تواند mean-ergodic باشد ولی WSS نباشد.
	\\
نادرست است. چرا که فرآیند‌های mean-ergodic زیر مجموعه‌ای از فرآیند‌های WSS هستند.
	\item
حاصل جمع دو فرآیند WSS همواره یک فرآیند WSS است.
	\\
نادرست است. برای اتوکورلیشن جمع دو فرآیند فرضی X و Y داریم:
$$
\mathbb{E}[(X(t) + Y(t))(X(S) + Y(S))] 
$$
$$
= \mathbb{E}[X(t)X(S)] + \mathbb{E}[Y(t)Y(S)] + \mathbb{E}[X(t)X(S)] + \mathbb{E}[X(s)X(t)]
$$
$$
= R_{X} (t-s) + R_Y(t-s) + R_{XY} (t,s) + 2R_{XY} (s,t) 
$$
که الزاما
$ R_{XY} $
تابعی از اختلاف زمان نیست.
\end{enumerate}
